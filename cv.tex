% Set document class and font size
\documentclass[letterpaper, 11pt]{article}
\usepackage[utf8]{inputenc}
% Package imports
\usepackage{fontawesome5, fancyhdr, longtable, paralist, paracol}
\usepackage[unicode, draft=false]{hyperref}
\usepackage[backend=biber, style=authortitle, sorting=ydnt, maxbibnames=999, url=false, isbn=false]{biblatex}
% sort bibliography by year, then date, then title in descending order
% Copy from https://tex.stackexchange.com/questions/484108/biblatex-sort-by-year-and-name-both-descending

\DeclareSortingTemplate{ydnt}{
  \sort{
    \field{presort}
  }
  \sort[final]{
    \field{sortkey}
  }
  \sort[direction=descending]{
    \field{sortyear}
    \field{year}
    \literal{9999}
  }
  \sort[direction=descending]{
    \field{sortname}
    \field{author}
    \field{editor}
    \field{translator}
    \field{sorttitle}
    \field{title}
  }
  \sort{
    \field{sorttitle}
    \field{title}
  }
}

\addbibresource{citations.bib}

% Make author name bold
\newcommand{\makeauthorbold}[1]{%
  \DeclareNameFormat{author}{%
    \ifthenelse{\value{listcount}=1}
    {%
      {\expandafter\ifstrequal\expandafter{\namepartfamily}{#1}{\mkbibbold{\namepartfamily\addcomma\addspace \namepartgiveni}}{\namepartfamily\addcomma\addspace \namepartgiveni}}
      %
    }{\ifnumless{\value{listcount}}{\value{liststop}}
        {\expandafter\ifstrequal\expandafter{\namepartfamily}{#1}{\mkbibbold{\addcomma\addspace \namepartfamily\addcomma\addspace \namepartgiveni}}{\addcomma\addspace \namepartfamily\addcomma\addspace \namepartgiveni}}
        {\expandafter\ifstrequal\expandafter{\namepartfamily}{#1}{\mkbibbold{\addcomma\addspace \namepartfamily\addcomma\addspace \namepartgiveni\addcomma\isdot}}{\addcomma\addspace \namepartfamily\addcomma\addspace \namepartgiveni\addcomma\isdot}}%
      }
    \ifthenelse{\value{listcount}<\value{liststop}}
    {\addcomma\space}{}
  }
}

\makeauthorbold{Gonzalez-Colin}

% Set page margins
\usepackage[left=0.5in, right=0.6in, bottom=0.7in, top=0.5in]{geometry}

\title{rewrite_cv}

% --- Document starts here ---
\begin{document}
\pagestyle{fancy}


% Set Footer
\fancyfoot{}
\fancyfoot[R]{Curriculum Vitae, Cristian Gonzalez-Colin, \thepage}

% Name and date of last update to this document
\noindent{\Huge{Cristian Gonzalez-Colin}\hfill\it{\footnotesize Updated \today}} \\[5pt]
\noindent{PhD Student, Bioinformatics and Systems Biology}\\

% Division line
\vspace{4pt}
\hrule
\vspace{-12pt}
% --- Start the two-column table storing the main content ---

% Set spacing between columns
\setlength{\tabcolsep}{8pt}

% Set the width of each column
\begin{longtable}{@{}p{1.1in}p{6.05in}}

% --- Contact information ---
{\sc Contact} &
\begin{tabular}{llll}
    \\
    % Line 1: Github, LinkedIn, Orcid, Twitter
    \faLinkedin \href{https://linkedin.com/in/cris-gonzalezcolin}{\it cris-gonzalezcolin} &
    \faGithub \href{https://github.com/cristian2420}{\it cristian2420} &
    \faOrcid \href{https://orcid.org/0000-0002-3920-3740}{\it 0000-0002-3920-3740} &
    \faTwitter \href{https://twitter.com/MeDicenCrix}{\it MeDicenCrix} \\
    % Line 2: Emails & employers
    \faEnvelope \href{mailto:cgonzalez@lji.org}{\it cgonzalez@lji.org} &
    \faBuilding \href{https://lji.org/}{\it LJI} &
    \faEnvelope \href{mailto:cgonzalezcolin@ucsd.edu}{\it cgonzalezcolin@ucsd.edu} &
    \faUniversity \href{https://ucsd.edu/}{\it UCSD} \\[3pt]
    \end{tabular}
    \\    
& \\
% --- Interests ---
\nohyphenation{{\sc Research Interest}} & Understanding the effect of genetic variants linked to human diseases in 
immune-related cell types, through the development of computational tools. \\ 
& \\

% --- Education ---
{\sc Education} &
% PhD
\textbf{University of California, San Diego} \hfill La Jolla, California \\
& PhD in \href{https://bioinformatics.ucsd.edu/}{Bioinformatics and Systems Biology} \hfill In Progress \\[2pt]


% Bachelor's
& \textbf{Center for Genomic Sciences, UNAM} \hfill Morelos, Mexico \\
& BS in Genomic Sciences \hfill 2016-2020 \\
& \begin{minipage}[c]{0.75\linewidth}
    \begin{compactitem}
        \item Dissertation: Effects of disease risk-variants in gene expression at single cell level
        \item Thesis Comittee: Pandurangan Vijayanand, MD, PhD; Benjamin Schmiedel, PhD; Yvonne Rosenstein, PhD.
        \item Global Average: 9.3 out of 10        
    \end{compactitem}
 \end{minipage}
\\[0.38in]
% Previous BS
& \textbf{Faculty of Sciences, UNAM} \hfill Mexico City, Mexico \\
& BS in Biology \hfill 2015-2016 \\
& \\

% --- Uncomment the next few lines if you want to include some courses ---
%& \textbf{Selected coursework}
%\begin{itemize}[noitemsep,leftmargin=*]
%\item \underline{Relevant subject 1}: Course 1, Course 2, Course 3, Course 4
%\item \underline{Relevant subject 2}: Course 1, Course 2, Course 3, Course 4
%\end{itemize} \\

% --- Research experience ---
{\sc Research} &
\textbf{PhD Student} \hfill 2022 -- Present   \\
{\sc Experience} & \textit{Vijayanand Lab, La Jolla Institute for Immunology} \hfill La Jolla, California \\
& Mentor: \href{https://www.lji.org/labs/vijayanand/}{Pandurangan Vijayanand, MD, PhD} \\
% --
&\textbf{La Jolla Institute for Immunology} \hfill 2019-2022\\
& \textit{Research Technician}\\
& \href{https://www.lji.org/labs/vijayanand/}{\textit{Vijayanand Lab, La Jolla Institute for Immunology}} \hfill La 
Jolla, California \\
& Quantitative Trait Loci (QTLs) for gene expression and histone \\ & marks in DICE database.  \\
% --
&\textbf{International Laboratory for Human Genome Research} \hfill 2018-2019\\
& \textit{Undergraduate Researcher}\\
& \href{https://liigh.unam.mx/amedina/}{\textit{Regulatory Genomics and Bioinformatics Lab}} \hfill Queretaro, Mexico \\
& Development of tools for the identification of conserved regulatory \\ & regions in Prokaryotes genomes for the RSAT 
suite tools. \\
% --
&\textbf{Center for Genomic Sciences, UNAM} \hfill 2017-2019\\
& \textit{Undergraduate Researcher}\\
& \href{https://www.ccg.unam.mx/en/computational-genomics/}{\textit{Computational Genomics Lab}} \hfill Morelos, Mexico 
\\
& Development of machine learning tools for the improvement and \\ & automatization of analysis and biocuration on the 
REGULONDB. \\
& \begin{minipage}[c]{0.75\linewidth}
    \begin{compactitem}
    \item \nohyphenation{Automatic summarization of transcription factors (TFs) properties from text literature.}
    \item Supervised learning and text mining to retrieve regulatory interactions in bacterial literature.
    \item Text mining to retrieve transporter-substrate interactions.  
    \end{compactitem}
 \end{minipage}
 \\[0.41in]
% --
&\textbf{INMEGEN} \hfill 2018-2019 \\
& \textit{Undergraduate Intership} \hfill 2015-2016\\
& 
\href{https://www.inmegen.gob.mx/investigacion/departamentos/departamento/?id=unidad-de-vinculacion-cientifica-facultad-de-medicina-inmegen}{\textit{Faculty 
of Medicine-INMEGEN}}\\
& Analysis of preterm birth genomic markers in Mexican population. \hfill Mexico City, Mexico \\ & Determination of 
cytokine concentration in preterm birth samples. \\
& \\
\end{longtable} 
\pagebreak
% --- Publications ---
\columnratio{0.18}
\begin{paracol}{2}
  {\sc Publications} 
  \switchcolumn
\nocite{*} 
\printbibliography[heading=none]
\end{paracol}

% --- Restart the two-column table storing the main content ---

% % Set spacing between columns
\setlength{\tabcolsep}{8pt}

% % Set the width of each column
\begin{longtable}{p{1.1in}p{6.05in}}

% --- Conferences ---
{\sc Conference} 
& \textbf{Poster presentation at Keystone Symposia: Gene Regulation:} \hfill Summer 2022 \\
{\sc Presentation} &  \textbf{From Emerging Technologies to New Models.} \\
& \textit{The cis-regulatory lanscape reveals cell type- and context-depedent} \\
& \textit{effects of disease-risk variants affecting human immune cell types.}  \\
& \\


% --- Awards and Honors ---
% TODO: add Awards
% TODO: add Honors
% {\sc Awards and Honors}} \\

% --- Talks and tutorials ---
% TODO: add invited talks
% TODO: add posters
% {\sc Talks and tutorials}} 
% & \textbf{Title of your most recent presentation} \hfill Month Year \\
% & Name of conference, workshop, seminar, venue, etc., or a description \\
% & \\

% & \textbf{Title of your second most recent presentation} \hfill Month Year \\
% & Name of conference, workshop, seminar, venue, etc., or a description \\
% & \\

% ---  Mentorhship ---
{\sc{Mentorship}}
& \textbf{Elizabeth Marquez-Gomez} \hfill 2021-2023 \\
& Undergraduate Student, UNAM \hfill La Jolla, California\\
& \textit{Vijayanand Lab, La Jolla Institute for Immunology}  \\
& \\

% --- Teching ---
{\sc Teaching}
& \textbf{Teaching Assistant} \hfill Spring 2019 \\
& \textit{Center for Genomic Sciences} \hfill Morelos, Mexico\\
& Bioinformatics Course. \\ & Professors: Julio Collado-Vides and Heladia Salgado \\
& \\

% --- Outreach ---
{\sc Outreach}
& \textbf{Volunteer} \hfill Summer 2023 \\
& \textit{Camp Connect Science Class} \\
& \\

% --- Certifications ---
{\sc Certifications}
& \textbf{Introduction to Deep learning, UAEM} \hfill  Fall 2015 \\
& \\

% --- Skills ---
{\sc Skills}
& \textbf{Programming Languages:} R, Python, Bash \\

& \textbf{Languages:}  English, Spanish\\
&\\

% ---  Professional society memberships ---
% --- Note: section title is spread over two lines ---

% {\sc Professional}
% & {\textbf{Name of professional society.}} \hfill Month Year -- Present \\
% {\sc memberships}
% & Some things you did or conferences you attended. Aliquam volutpat est vel massa. Sed dolor lacus, imperdiet non, 
ornare non, commodo eu, neque. \\
% & \\

% --- Other interests/hobbies ---

% \nohyphens{\sc Other interests} & Some of your hobbies etc.\\

% --- End of CV! ---

\end{longtable}
\end{document}

